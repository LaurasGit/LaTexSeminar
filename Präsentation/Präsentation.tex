\documentclass[11pt]{beamer}
\usetheme[progressbar=frametitle]{metropolis}
\usepackage{xcolor} %Farbe überschriften
\definecolor{blue}{RGB}{24,116,190}
\setbeamercolor{progress bar}{fg=blue}

%\usecolortheme{owl}

\usepackage{appendixnumberbeamer}
\usepackage[utf8]{inputenc}
\usepackage[T1]{fontenc}
\usepackage{graphicx}
\usepackage{lmodern}
%\usepackage{beamercolorthemeowl}
\usepackage{pgfpages}
%\setbeameroption{show notes} %Notizen anzeigen
%\setbeameroption{show notes on second screen}
\makeatletter
\setlength{\metropolis@progressinheadfoot@linewidth}{3pt}
\setlength{\metropolis@titleseparator@linewidth}{1pt}
\setlength{\metropolis@progressonsectionpage@linewidth}{3pt}

\author{Laura Hartzheim}

\title{Arten des Machine Learnings - Supervised, Unsupervised und Reinforcement Learning}

\subtitle{}

\logo{}

\institute{}

\date{}

\subject{}

\setbeamercovered{transparent}

\setbeamertemplate{navigation symbols}{}

\begin{document}
	
	
	\begin{frame}
		\titlepage
	\end{frame} 
	
	\begin{frame}
		\frametitle{Inhalt}
		\tableofcontents
	\end{frame} 
	
	\section{Machine Learning}
	\begin{frame}
		\frametitle{Was ist Machine Learning?}
		\begin{itemize}
			\item Schnittmenge aus Statistik, Künstlicher Intelligenz und Informatik
			\item Maschine soll aus Daten lernen können, durch Erfahrung und Leistungsmessung
			%Bsp spam filter
		\end{itemize} 
	\end{frame}
	
	\begin{frame}
		\frametitle{Warum nutzt man Machine Learning?}
		\begin{itemize}
			\item vereinfachter Code und bessere Performanz bei Problemen mit vielen Regeln, da Regeln von Machine gelernt werden 
			\item Programme leichter zu warten und weniger fehleranfällig
			\item bietet Lösungen für komplexe Probleme die durch normale Programme nicht lösbar sind
			\item und vieles mehr
			%Email bsp
			
		\end{itemize} 
	\end{frame}
	
	\section{Supervised Learning}
	
	\begin{frame}
		
	\end{frame}
	
	\section{Unsupervised Learning}
	
	\begin{frame}
		
	\end{frame}
	
	\section{Reinforcement Learning}
	
	\begin{frame}
		
	\end{frame}
	

\end{document}