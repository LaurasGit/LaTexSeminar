% !TEX root = Seminararbeit.tex
Machine Learning ist ein wesentlicher Teil von vielen kommerziellen Anwendungen und Forschungsprojekten. \cite{Mueller2016} Es wird für Gesichtserkennung und Handschrifterkennung verwendet \cite{Kirk2014}, hat aber auch Einzug in noch alltäglichere Dinge, wie Film-, Essens- und Produktvorschläge genommen. Machine Learning ist weit verbreitet. Es wird zum Beispiel in der medizinischen Diagnose und Behandlung, sowie im Finden von Freunden in Sozialen Netzen genutzt. \cite{Mueller2016} Die erste weitverbreitete Anwendung ist der Spam-Filter, welcher bereits in den Neunzigern entwickelt wurde. \cite{Geron2017}\newline
	\textbf{Aber was ist Machine Learning eigentlich?}\newline
	Es ist die Schnittmenge aus Statistik, Künstlicher Intelligenz und Informatik. \cite{Mueller2016} Beim Machine Learning soll der Computer aus Daten lernen können, ohne explizit für das Lernziel programmiert zu sein. Das Programm soll also aus Erfahrung, bezüglich einer bestimmten Aufgabe und Leistungsmessung, lernen. Die erzielte Leistung soll sich mit der Erfahrung verbessern.
	Am Beispiel des Spam-Filters würde das also bedeuten, dass das Programm aus vorgegebenen Spam-E-Mails, die zum Beispiel von Usern markiert wurden, und normalen E-Mails lernt. Die zum Lernen verwendeten Daten, welche in diesem Fall E-Mails sind, nennt man Trainings-Daten. In diesem Fall wäre die Aufgabe Spam-Mails zu flaggen, die Erfahrung lässt sich aus den Trainings-Daten ableiten und die Leistung könnte zum Beispiel an der Anzahl richtig geflaggter E-Mails gemessen werden. \cite{Geron2017} \newline
	\textbf{Warum wird Machine Learning eingesetzt?}\newline
	Es bietet Vorteile bei Problemen mit vielen Regeln. Der Machine Learning Algorithmus kann den Code wesentlich vereinfachen und führt zu einer besseren Performanz gegenüber herkömmlichen Programmen. Würde man den Spam-Filter ohne Machine Learning programmieren, müssten Regeln für die  Absender-E-Mail-Adresse, in Spam-Mails oft vorkommenden Begriffe wie Kredikarte und kostenlos und vieles mehr programmiert werden. Diese würden zu einem sehr langen Code führen. Der Machine Learning Algorithmus lernt diese Regeln selbst. Ein kurzes Programm ist leichter zu pflegen und weniger weniger anfällig für Fehler.\newline
	Bei komplexen Problemen gibt es teilweise mit traditionellen Algorithmen noch keine oder keine gute Lösung. Machine Learning kann diese Probleme lösen.\newline
	Verändert sich die Umgebung eines Problems kann sich ein Machine Learning System mit Hilfe von neuen Daten an die Umgebung anpassen.\newline
	Durch die Verwendung von Machine Learning können auch neue Erkenntnisse in großen Datenmengen und komplexen Problemen gefunden werden. \cite{Geron2017}\newline
	Das Ziel von Machine Learning ist es, Daten in etwas Bedeutsames zu manipulieren, was generell immer wichtiger wird. \cite{Kirk2014} Um einen besseren Einblick in dieses Thema zu gewinnen, werden in dieser Seminararbeit die grundlegenden Arten des maschinellen Lernens beschrieben und schließlich verglichen.