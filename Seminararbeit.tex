\documentclass[12pt,titlepage]{article}

%pakete
\usepackage[ngerman]{babel}
\usepackage[latin1]{inputenc}
\usepackage{color}
\usepackage[a4paper,lmargin={4cm},rmargin={2cm},
tmargin={2.5cm},bmargin = {2.5cm}]{geometry}
\usepackage{amssymb}
\usepackage{amsthm}
\usepackage{graphicx}

\linespread{1.25} %Zeilenabstand


\begin{document}
	
	%Titelseite
	\begin{titlepage}
		\title{Arten des Machine Learnings - Supervised, Unsupervised und Reinforcement Learning}
		\date{2018}
		\author{Laura Hartzheim}
		\maketitle
	\end{titlepage}
	
	%Inhaltsverzeichnis
	\tableofcontents
	\newpage
	\setcounter{page}{1}
	
	\section{Einleitung}
	\newpage
	\section{Supervised Learning}
	Supervised Learning geh�rt zu den erfolgreichsten und meist verbreiteten Arten des Machine Learnings. (M�ller, S.25)
	Beim Supervised Learning werden bekannte Daten und Ausgaben w�hrend dem Trainieren und Pr�fen des Models genutzt, welche auch Training-Daten und Label genannt werden.(Sarkar, S.35) Diese optimieren das Model, auf Basis der Vorhandenen Daten, durch anpassen der Parameter.(Suthaharan, S.140) Ein Model besteht aus den in- und output-Paaren des Training Datensatzes. (M�ller, S25) Das Hauptziel ist es die eingehenden Daten x auf die ausgehenden y Abzubilden(f(x) = y), um sp�ter f�r neue Daten x' die zugeh�rigen y' Daten zu bestimmen.(Sarkar, S.35)
	Durch eine gr��ere Menge an Traning-Daten ist eine bessere Abdeckung von Verschiedenen F�llen m�glich, dies kann aber auch zu Overfitting f�hren.Um das zu verhindern muss das Training fr�h genug beendet werden.(Suthaharan, S.140)
	Es gibt zwei Methoden f�r Supervised Learning, Klassifikation und Regression. Die Wahl der Methode h�ngt von der zu erf�llenden Aufgabe ab.(Sarkar, S.35)
	
	\subsection{Klassifikation}
	
	Das Ziel der Klassifikation ist es ein Klassenlabel f�r die eingehenden Daten voraus zusagen. Die verschiedenen Label sind Teil einer vorgegebenen Liste. (M�ller, S.25) Die Klassifikation kann in bin�re und multiklassen Klassifkation aufgeteilt werden. Bei bin�rer Klassifikation sind nur zwei Klassen verf�gbar, die Problemstellung l�sst sich also auf eine Ja/Nein-Frage ableiten. In der multiklassen Klassifikation sind mehrere Klassen m�glich.(M�ller, S.25) 
	
	\subsection{Regression}
	

	\newpage
	\section{Unsupervised Learning}
	\newpage
	\section{Reinforcement Learning}
	\newpage
	\section{Schluss}
	
\end{document}