\documentclass[12pt,titlepage]{article}

%pakete
\usepackage[ngerman]{babel}
\usepackage[latin1]{inputenc}
\usepackage{color}
\usepackage[a4paper,lmargin={4cm},rmargin={2cm},
tmargin={2.5cm},bmargin = {2.5cm}]{geometry}
\usepackage{amssymb}
\usepackage{amsthm}
\usepackage{graphicx}

\linespread{1.25} %Zeilenabstand


\begin{document}
	
	%Titelseite
	\begin{titlepage}
		\title{Arten des Machine Learnings - Supervised, Unsupervised und Reinforcement Learning}
		\date{2018}
		\author{Laura Hartzheim}
		\maketitle
	\end{titlepage}
	
	%Inhaltsverzeichnis
	\tableofcontents
	\newpage
	\setcounter{page}{1}
	
	\section{Einleitung}
	\newpage
	\section{Supervised Learning}
	Supervised Learning geh�rt zu den erfolgreichsten und meist verbreiteten Arten des Machine Learnings. (M�ller, S.25)
	Beim Supervised Learning werden bekannte Daten und Ausgaben w�hrend dem Trainieren und Pr�fen des Models genutzt, welche auch Training-Daten und Label genannt werden.(Sarkar, S.35) Diese optimieren das Model, auf Basis der Vorhandenen Daten, durch anpassen der Parameter.(Suthaharan, S.140) Ein Model besteht aus den in- und output-Paaren des Training Datensatzes. (M�ller, S25) Das Hauptziel ist es die eingehenden Daten x auf die ausgehenden y Abzubilden(f(x) = y), um sp�ter f�r neue Daten x' die zugeh�rigen y' Daten zu bestimmen.(Sarkar, S.35)
	Durch eine gr��ere Menge an Traning-Daten ist eine bessere Abdeckung von Verschiedenen F�llen m�glich, dies kann aber auch zu Overfitting f�hren.Um das zu verhindern muss das Training fr�h genug beendet werden.(Suthaharan, S.140)
	Es gibt zwei Methoden f�r Supervised Learning, Klassifikation und Regression. Die Wahl der Methode h�ngt von der zu erf�llenden Aufgabe ab.(Sarkar, S.35)
	
	\subsection{Klassifikation}
	
	Das Ziel der Klassifikation ist es ein Klassenlabel f�r die eingehenden Daten voraus zusagen. Die verschiedenen Label sind Teil einer vorgegebenen Liste. (M�ller, S.25) Die Klassifikation kann in bin�re und multiklassen Klassifkation aufgeteilt werden. Bei bin�rer Klassifikation sind nur zwei Klassen verf�gbar, die Problemstellung l�sst sich also auf eine Ja/Nein-Frage ableiten. In der multiklassen Klassifikation sind mehrere Klassen m�glich.(M�ller, S.25) W�hrend der Training-Phase werden Regeln f�r das Zuteilen von Labels erstellt, die sp�ter dabei helfen Test-Daten Labels zu zuweisen. (Suthaharan, S.8)
	
	
	\subsection{Regression}
	In Regression-Problemen sollen oft Zahlen oder Werte ermittelt werden. Im Gegensatz zur Klassifikation gibt es keine Klassen oder Labels denen Daten zugeordnet werden k�nnen. Regression-Modelle lernen den Zusammenhang aus Eingangs- und Ausgangsdaten um f�r neue Daten den passenden Output vorherzusagen.(Sarkar, S.37) \newline
	Simple linear regression-Modelle versuchen mit nur einem Feature der einer Variable x eine Output-Variable y zu bestimmen und k�nnen somit lineare Probleme l�sen.(Sarkar, S.37)\newline
	Multivariable Regressions Methoden werden f�r Probleme mit mehreren input-Variablen in Form eines Vektors und nur einer Output-Variable verwendet.(Sarkar, S.38) \newline
	Ein Sonderfall der Multivariablen Regression ist die Polynomiale Regression. Hier ist die Ausgabevariable Polynom n-ten Grades der Eingangsvariable  (Sarkar, S.38)\newline
	Nichtlineare Regressions Modelle stellen zwischen Ein- und Ausgehendendaten eine Beziehung auf Basis einer Kombination aus nicht-linearen Funktionen her.(Sarkar, S.38)\newline
	
	...Verschiedene Regr.Models laut Sarkar S37/38:...
	(Sarkar, S.38)
	Lasso regression is a special form of regression that performs normal regression and generalizes the
	model well by performing regularization as well as feature or variable selection. Lasso stands for least absolute
	shrinkage and selection operator. The L1 norm is typically used as the regularization term in lasso regression
	\newpage
	\section{Unsupervised Learning}
	\newpage
	\section{Reinforcement Learning}
	\newpage
	\section{Schluss}
	
\end{document}